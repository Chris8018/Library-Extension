\documentclass[12pt,a4paper]{article}
\title{Case-Study: Modelling a Library System}
%\author{}
\date{}

\usepackage{oofs}

\addtolength{\textheight}{10ex}

\setlength{\unitlength}{1mm} % for pictures

\newcounter{exer} % for exercises
\renewcommand{\thefootnote}{\fnsymbol{footnote}} % symbols
\newcommand{\exer} % ref exercise (by a marginal note)
	{\stepcounter{exer}{\;\arabic{exer}}}

\begin{document}\maketitle

\vspace{-1cm}
\noindent This case-study involves modelling an information system to support a simple
\emph{Lending Library Service}. This will be approached as an exercise in
\emph{Requirements Engineering}
where the aim is to develop a \emph{model}, capturing
high-level requirements for such a system
to produce an agreed \emph{formal specification}.
We make use of \emph{Formal Methods},
exploiting the power of discrete mathematics
to model, and to reason, \emph{abstractly} over a succession of separate \emph{levels}
producing a \emph{modular} de-composition of the system.



\subsubsection*{Membership Class: $\i{LM}$}

We begin by defining an invariant for a simple membership class. Members belong to a finite, initially-empty,
set named \i{Member} and each member has some associated \emph{information}:
\begin{showspecs}
	\begin{spec}{\i{LM}}
		\i{Member}: \p{set}\;\g{M}\\
		\i{info}: \i{Member} \tfun \g{I}
	\post	\i{Member'} = \emptyset
	\end{spec}
\end{showspecs}

\noindent
The \emph{types} for both a member identifier and its
associated information are specified here as two given, \emph{fully-abstract}, sets named \g{M} and \g{I};
the definition of their concrete data-representations is left to later \emph{refinements}.
The \emph{initial-condition} for class \i{LM},
together with its \emph{invariant}, imply that all components
are empty when the class is first \emph{instantiated}.

\medskip\noindent A \emph{query} is an operation that provides information about the state of objects in the system.
In this class, we can define a query that shows the information associated with a member:
\begin{showspecs}
	\begin{spec}{\i{LM?showInfo(m \to i)}}
		i := \i{info}(m)
	\end{spec}
\showbeside
	\begin{spec}[\equiv]{\i{LM?showInfo(m \to i)}}
		m:\i{Member} \sep i: \g{I}\\ i = \i{info}(m)
	\end{spec}
\end{showspecs}
\noindent and another that lists all the members, together with their information:
\begin{showspecs}
	\begin{spec}{\i{LM?showMembers(\to M)}}
		M:= \i{info}
	\end{spec}
\showbeside
	\begin{spec}[\equiv]{\i{LM?showMembers(\to M)}}
		M: \i{Member} \tfun \g{I} \sep M = \i{info}
	\end{spec}
\end{showspecs}
\noindent In both cases their \emph{pre-conditions} have been expanded
to clarify these specifications by providing type information implicit from the invariant.

\medskip\noindent An \emph{event} is an operation that changes the state of the system. We obviously require at least one event
for this class, an operation to `enrol' a new member, since the set \i{Member} is initially empty:
\begin{showspecs}
	\begin{spec}{\i{LM!NewMember(i \to m)}}
		i: \g{I} \sep m: \g{M} \sep m\notin\i{Member}
	\post	m \in \i{Member'} \sep \i{info'}(m) =\, i
	\end{spec}
\end{showspecs}

\smallskip\noindent An operation to `update' the recorded
information for existing members can also be defined:
\begin{showspecs}
	\begin{spec}{\i{LM!UpdateInfo(m,i)}}
		m: \i{Member} \sep i: \g{I} \sep i \neq \i{info}(m)
	\post	\i{info'}(m) =\, i
	\end{spec}
\end{showspecs}
\smallskip\noindent \textbf{Exercise}\exer\\
Provide a \emph{simple}
specification of an operation to `remove' a current member.


\subsubsection*{Catalog Class: $\i{LT}$}

The library's catalog can be modelled as an
initially-empty set of \emph{titles},
where each title has an associated \emph{description}:
\begin{showspecs}
	\begin{spec}{\i{LT}}
		\i{Title}: \p{set}\;\g{T}\\
		\i{desc}: \i{Title} \tfun \g{D}
	\post	\i{Title'} = \emptyset
	\end{spec}
\end{showspecs}
\noindent Observe that class \i{LT} has a
\emph{specification} that is very similar to the class \i{LM} -- only the \emph{names}
differ. The operations we need for \i{LT} are also similar to those provided for \i{LM}.

\medskip\noindent We can provide a query that shows the description of a particular title:
\begin{showspecs}
	\begin{spec}{\i{LT?showDesc(t \to d)}}
		d := \i{desc}(t)
	\end{spec}
\end{showspecs}
\noindent and one that lists the full catalog of titles together with their descriptions:
\begin{showspecs}
	\begin{spec}{\i{LT?showCatalog(\to T)}}
		T:= \i{desc}
	\end{spec}
\end{showspecs}

\newpage

\medskip\noindent We require at least one event for \i{LT}
-- an operation to `enter' a new title into the system, since the set \i{Title} is initially empty:
\begin{showspecs}
	\begin{spec}{\i{LT!NewTitle(d \to t)}}
		d:\g{D} \sep t: \g{T} \sep t \notin\i{Title}
	\post	t \in \i{Title'} \sep \i{desc'}(t) =\, d
	\end{spec}
\end{showspecs}
\medskip\noindent
\textbf{Exercise}\exer\\
Suppose that an operation to `update' the description for
a known title along with one to `remove' such
a title were required at the level of class
\i{LT}. Give a \emph{simple} specification for each event.


%\newpage\defwidth[5.75cm]
\subsubsection*{Collection Class: $\i{LC}$}

We model a library's `collection' (of hard copies) by \emph{extending} the catalog class.
We add a function \emph{nc} to record the current `number of copies'
for every title and define a \emph{partition} over the set of such titles, based on the value of \emph{nc};
titles are \emph{in-collection} if there are copies of the title in the collection,
otherwise they are \emph{ex-collection}:
\begin{showspecs}
	\begin{spec}{\i{LC}}
		\i{LT}\\[0.5ex]
		\i{nc}: \i{Title} \tfun \p{NAT}\\[0.5ex]
		\{\i{InColl,ExColl}\}: \p{part}\,\i{Title}\\
		\i{InColl}:=\!\{t\!:\!\i{Title}\bullet\i{nc}(t) > 0\}
	\end{spec}
\showbeside
	\begin{spec}[\equiv]{\i{LC}}
		\i{Title}: \p{set}\;\g{T}\\
		\i{desc}: \i{Title} \tfun \g{D}\\
		\i{nc}: \i{Title} \tfun \p{NAT}\\
		\i{InColl}:=\!\{t\!:\!\i{Title}\bullet\i{nc}(t) > 0\}\\
		\i{ExColl}\!:=\!\{t\!:\!\i{Title}\bullet\i{nc}(t) = 0\}
	\post	\i{Title'} = \emptyset
	\end{spec}
\end{showspecs}
\noindent Other state-components, and the initial-condition, are
``inherited" from \i{LT}. That (intermediate) class specification
is said to be \emph{imported} into class \i{LC}: its invariant must
always hold there, but this may be consistently \emph{extended};
the operations it provides may also be applied or extended
in such contexts.

\smallskip\noindent Required queries at this level might be similar
to those provided for a catalog.
Indeed, we will directly \emph{promote} the query \emph{showDesc} from class \i{LT}:
\begin{showspecs}
	\begin{spec}{\i{LC?showDesc(t \to d)}}
		\i{LT?showDesc(t \to d)}
	\end{spec}
\end{showspecs}
\noindent and add a query showing the number of copies for a given title:
\begin{showspecs}
	\begin{spec}{\i{LC?showNoCopies(t \to n)}}
		n:=\i{nc}(t)
	\end{spec}
\end{showspecs}

\newpage
\noindent However, we will redefine the query \emph{showCatalog} to include
information only about titles that are in-collection and extend the query to include the
number of copies for each title as well as its description:
\begin{showspecs}
	\begin{spec}{\i{LC?showCollection(\to C)}}
		C: \i{Title}\pfun\p{POS}\!\times\g{D}\\
		C\!=\!\{(t,n,d):\i{InColl}\times\p{POS}\!\times\g{D}\bullet n = \i{nc}(t)\wedge d = \i{desc}(t)\}
	\end{spec}
\end{showspecs}


\smallskip\noindent We extend the \i{NewTitle} event from \i{LT},
so as to `fix' (as zero) an \emph{initial} number of copies:
\begin{showspecs}
	\begin{spec}{\i{LC!NewTitle(d \to t)}}
		\i{LT!NewTitle(d \to t)}
	\post	\i{nc'}(t) =\,0
	\end{spec}
\showbeside
	\begin{spec}[\equiv]{\i{LC!NewTitle(d \to t)}}
		d:\g{D} \sep t:\g{T} \sep t\notin\i{Title}
	\post	t \in \i{Title'\,\cap ExColl'}\\
		\i{Desc'}(t) =\,d \sep \i{nc'}(t) =\,0
	\end{spec}
\end{showspecs}\smallskip\noindent
\noindent When a new title is added it is ex-collection.
We then clearly need some way to `modify' the number
of copies as well
\begin{showspecs}
	\begin{spec}{\i{LC!AddCopies(t,n)}}	
		t: \i{Title} \sep n: \p{POS}
	\post	\i{nc'}(t) = \i{nc}(t) + n
	\end{spec}
\showbeside
	\begin{spec}{\i{LC!RemoveCopy(t)}}	
		t: \i{InColl}
	\post	\i{nc'}(t) = \i{nc}(t) - 1
	\end{spec}
\end{showspecs}\medskip\noindent
The operations specified at this level involve modelling choices.
These decisions should satisfy the relevant requirements
of our intended \emph{client}.

\medskip\noindent
\textbf{Exercise}\exer\\
Expand the \i{LC} events \i{AddCopies} and \i{RemoveCopy} to show all
\emph{changes-of-state} imposed by the state-invariant.

\medskip\noindent
\textbf{Exercise}\exer\\
Suppose that operations for class \i{LC} do not fully
satisfy requirements of a client. Explore some possible
alternatives, then specify that query or event:

\vspace{-1ex}\begin{enumerate}
\item there should be a way to show descriptions for titles
	having \emph{no} copies;

\item it should be possible to fix the initial number of copies
	for a new title;

\item it should be possible to remove \emph{more than one} copy at the
	same time;

\item one operation should serve to \emph{increase} or \emph{decrease}
	the number of copies.
\end{enumerate}


\newpage
\subsubsection*{Loan Class:
		$\i{LL}$}

We model a simple loan class by \emph{composing} the
classes \i{LM} and \i{LC}. These two classes are independent so composing them will produce a consistent invariant.
We then extend their combined state,
to define the set of all current `loans', which is expressed mathematically as a
\emph{relation} (\i{loan}), and the numbers of `available'
and `loaned' copies (the functions \i{na} and \i{nl}, respectively) for each title:
\begin{showspecs}
	\begin{spec}{\i{LL}}
		\i{LM} \sep \i{LC}\\[0.5ex]
		\i{loan}: \i{Title \prel Member}\\[0.5ex]
		\i{na,nl}: \i{Title} \tfun \p{NAT}\\[0.5ex]
		\forall t: \i{Title}\, \bullet \\
        \hspace{0.5cm}\i{nc}(t) = \i{na}(t)+\i{nl}(t) \,\wedge\\
		\hspace{0.5cm}\i{nl}(t) = \#\{m:\i{Member}\bullet t\mapsto m\in \i{loan}\}
	\end{spec}
\end{showspecs}

\begin{showspecs}
	\begin{spec}[\equiv]{\i{LL}}
		\i{Member}: \p{set}\;\g{M}\\
		\i{info}: \i{Member} \tfun \g{I}\\[0.5ex]
		\i{Title}: \p{set}\;\g{T}\\
		\i{desc}: \i{Title} \tfun \g{D}\\[0.5ex]
		\i{loan}: \i{Title \prel Member}\\
		\i{nc,na,nl}: \i{Title} \tfun \p{NAT}\\[0.5ex]
		\i{InColl}:=\!\{t\!:\!\i{Title}\bullet\i{nc}(t) > 0\}\\
		\i{ExColl}\!:=\!\{t\!:\!\i{Title}\bullet\i{nc}(t) = 0\}\\[0.5ex]
		\forall t: \i{Title}\, \bullet \\
        \hspace{0.5cm}\i{nc}(t) = \i{na}(t)+\i{nl}(t) \,\wedge\\
		\hspace{0.5cm}\i{nl}(t) = \#\{m:\i{Member}\bullet t\mapsto m\in \i{loan}\}
	\post	\i{Member'} = \emptyset \sep \i{Title'} = \emptyset
	\end{spec}
\end{showspecs}

\medskip\noindent Class \i{LL} embodies a significant assumption:
\emph{a member may borrow at most one copy of any particular title at the same time}.

\bigskip\noindent The membership queries are needed at this level so they are
directly \emph{promoted}, as are two of the collection queries:
\begin{showspecs}
	\begin{spec}{\i{LL?showInfo(m \to i)}}
		\i{LM?showInfo(m \to i)}
	\end{spec}
\showbeside
	\begin{spec}{\i{LL?showMembers(\to M)}}
		\i{LM?showMembers(\to M)}
	\end{spec}
\end{showspecs}\smallskip
\begin{showspecs}
	\begin{spec}{\i{LL?showDesc(t \to d)}}
		\i{LC?showDesc(t \to d)}
	\end{spec}
\showbeside
	\begin{spec}{\i{LL?showNoCopies(t \to n)}}
		\i{LC?showNoCopies(t \to n)}
	\end{spec}
\end{showspecs}\noindent

\medskip\noindent
We add a query to show the number of available copies for a given title:
\begin{showspecs}
	\begin{spec}{\i{LL?availableCopies(t \to n)}}
		n:=\i{na}(t)
	\end{spec}
\end{showspecs}

\medskip\noindent
and update the \i{showCollection} query from the collection class:
\begin{showspecs}
	\begin{spec}{\i{LL?showCollection(\to C)}}
		C: \i{Title}\pfun\p{POS}\!\times\p{NAT}\!\times\g{D}\\
		C\!=\!\{(t,n,l,d):\i{InColl}\times\p{POS}\!\times\p{NAT}\!\times\g{D}\;\bullet\\
        \hspace{1cm} n = \i{nc}(t)\wedge l = \i{nl}(t)\wedge d = \i{desc}(t)\}
	\end{spec}
\end{showspecs}

\medskip\noindent
Finally, we add a query that shows the set of members borrowing copies of a given title:
\begin{showspecs}
	\begin{spec}{\i{LL?showLoans(t \to M)}}
		M :=\{m:\i{Member}\,\bullet\, t\mapsto m\in \i{loan}\}
	\end{spec}
\end{showspecs}

\medskip\noindent
Many of the events at this level are also promotions:
\begin{showspecs}
	\begin{spec}{\i{LL!NewMember(i \to m)}}
		\i{LM!NewMember(i \to m)}
	\end{spec}
\showbeside
	\begin{spec}{\i{LL!UpdateInfo(m,i)}}
		\i{LM!UpdateInfo(m,i)}
	\end{spec}
\end{showspecs}\smallskip
\begin{showspecs}
	\begin{spec}{\i{LL!NewTitle(d \to t)}}
		\i{LC!NewTitle(d \to t)}
	\end{spec}
\showbeside
	\begin{spec}[\equiv]{\i{LL!NewTitle(d \to t)}}
		d:\g{D} \sep t:\g{T} \sep t \notin \i{Title}
	\post	t \in \!\i{Title' \cap ExColl'};~ \i{desc'}(t) = d\\
		\i{nc'}(t) = \i{na'}(t) = \i{nl'}(t) = 0
	\end{spec}
\end{showspecs}

\medskip\noindent The \i{NewTitle} event has been expanded to show that the number of available and loaned copies is initially zero.
\i{AddCopies} is also promoted but expanded to show how the number of available copies is increased:
\begin{showspecs}
	\begin{spec}{\i{LL!AddCopies(t,n)}}	
		\i{LC!AddCopies(t,n)}
	\end{spec}
\showbeside
	\begin{spec}[\equiv]{\i{LL!AddCopies(t,n)}}	
		t: \i{Title} \sep n: \p{POS}
	\post	
		\i{nc'}(t) = \i{nc}(t) + n\\
		\i{na'}(t) = \i{na}(t) + n
	\end{spec}
\end{showspecs}

\medskip\noindent \i{RemoveCopy} is extended to clarify the pre-condition; the removed copy must be an available copy:
\begin{showspecs}
	\begin{spec}{\i{LL!RemoveCopy(t)}}	
		\i{LC!RemoveCopy(t)}\\
		\i{na}(t) > 0
	\end{spec}
\showbeside
	\begin{spec}[\equiv]{\i{LL!RemoveCopy(t)}}	
		t: \i{Title} \sep \i{nc}(t) \geq \i{na}(t) > 0
	\post	
		\i{nc'}(t) = \i{nc}(t) - 1\\
		\i{na'}(t) = \i{na}(t) - 1
	\end{spec}
\end{showspecs}

\medskip\noindent
We require an event to allow members to make loans:
\begin{showspecs}
	\begin{spec}{\i{LL!LoanCopy(t,m)}}
		t:\i{Title} \sep m: \i{Member}\\
		t\mapsto m \notin \i{loan}\\
		\i{na}(t) > 0
	\post	t\mapsto m \in \i{loan'}
	\end{spec}
\showbeside
	\begin{spec}[\equiv]{\i{LL!LoanCopy(t,m)}}
		t:\i{Title} \sep m: \i{Member}\\
		t\mapsto m \notin \i{loan}\\
		\i{na}(t) > 0
	\post	t\mapsto m \in \i{loan'}\\
		\i{na'}(t) = \i{na}(t) - 1\\
		\i{nl'}(t) = \i{nl}(t) + 1
	\end{spec}
\end{showspecs}

\medskip\noindent Finally at this level, we define an event to return a copy:
\begin{showspecs}
	\begin{spec}{\i{LL!Return(t,m)}}
		t\mapsto m: \i{loan}
	\post	t\mapsto m \notin \i{loan'}
	\end{spec}
\showbeside
	\begin{spec}[\equiv]{\i{LL!Return(t,m)}}
		t:\i{Title} \sep m: \i{Member}\\
		t\mapsto m \in \i{loan}\\
        \i{nl}(t) > 0
	\post	t\mapsto m \notin \i{loan'}\\
		\i{na'}(t) = \i{na}(t) + 1\\
		\i{nl'}(t) = \i{nl}(t) - 1
	\end{spec}
\end{showspecs}

\smallskip\noindent The expanded versions of \i{LoanCopy} and \i{Return}
show how the numbers of available and loaned copies change.

\smallskip\noindent
Throughout this specification we have defined \emph{minimal} versions of the operations.
We are able to do this because of the assumption that \emph{``the rest remains unchanged"}.
In some cases we used expansion to explicitly present interesting derived properties.

\medskip\noindent
\textbf{Exercise}\exer\\
Show that the assumption \emph{a member may borrow at most one copy of any particular title at the same time}
is implicit from the invariant of class \i{LL}.

\medskip\noindent
\textbf{Exercise}\exer\\
Define a query to show titles borrowed by a member.

\medskip\noindent
\textbf{Exercise}\exer\\
Suppose that operations to `remove' members or titles were provided for class \i{LL}.
Specify these events, which are \emph{applicable} only when the respective member or title does not figure in any current loan.
%
\newpage
\subsubsection*{Reservation Class: $\i{LR}$}

We specify \i{LR} as a class parallel to \i{LL} to allow titles to be reserved.
Eventually we will compose \i{LL} and \i{LR} to specify a library system able to provide both loans and reservations.
To specify \i{LR}, we again export and compose \i{LM} and \i{LC} as we did for \i{LL}.
We define a `request queue' (\i{requestQ}) to be a function that delivers,
for each title $t$, the sequence of members currently requesting a copy of $t$.
Similarly to \i{loans} in class \i(LL), we define the current reservations as a relation (\i{reserve}).
We also define the number of requests (\i{nQ}) for each title.

\begin{showspecs}
	\begin{spec}{\i{LR}}
		\i{LM} \sep \i{LC}\\[0.5ex]
		\i{requestQ}: \i{Title}\tfun\p{inj}\cdot\p{seq}\;\i{Member}\\[0.5ex]
		\i{reserve}: \i{Title} \prel \i{Member}\\
		\,\p{cf}\,\i{reserve} = \p{cod}\circ\i{requestQ}\\
		\i{nQ}:= (\#)\circ\i{requestQ}
	\end{spec}
\end{showspecs}
\begin{showspecs}
	\begin{spec}[\equiv]{\i{LR}}
		\i{Member}: \p{set}\;\g{M}\\
		\i{info}: \i{Member} \tfun \g{I}\\[0.5ex]
		\i{Title}: \p{set}\;\g{T}\\
		\i{desc}: \i{Title} \tfun \g{D}\\[0.5ex]
		\i{requestQ}: \i{Title}\tfun\p{inj}\cdot\p{seq}\;\i{Member}\\
		\i{reserve}: \i{Title} \prel \i{Member}\\
		\i{nc,nQ}: \i{Title} \tfun \p{NAT}\\[0.5ex]
		\i{InColl}:=\!\{t\!:\!\i{Title}\bullet\i{nc}(t) > 0\}\\
		\i{ExColl}\!:=\!\{t\!:\!\i{Title}\bullet\i{nc}(t) = 0\}\\[0.5ex]
		\forall t: \i{Title}\, \bullet \\
        \hspace{0.5cm}\{m:\i{Member}\bullet t\mapsto m\in \i{reserve}\} = \{m:\i{Member}\bullet m\in \p{cod}\,\i{requestQ}(t)\}\,\wedge\\
		\hspace{0.5cm}\i{nQ}(t) = \#\{m:\i{Member}\bullet t\mapsto m\in \i{reserve}\}\,\wedge\\
        \hspace{0.5cm}\i{nQ}(t) = \#\i{requestQ}(t)
	\post	\i{Member'} = \emptyset \sep \i{Title'} = \emptyset
	\end{spec}
\end{showspecs}

\medskip\noindent Class \i{LR} embodies another significant assumption:
\emph{a member may have at most one request for every title outstanding at the same time.}\\

\noindent As we plan to compose this class with \i{LL}, we won't promote or extend any of the existing queries to this level.
However, we will define a query to show the queue of requests for a particular title:

\begin{showspecs}
	\begin{spec}{\i{LR?showRequests(t \to Q)}}
		Q := \i{requestQ}(t)
	\end{spec}
\end{showspecs}

\noindent We need to extend the collection event
\i{NewTitle} at this level, to create a new (initially-empty)
request queue for the title that is introduced:
\begin{showspecs}
	\begin{spec}{\i{LR!NewTitle(d \to t)}}
		\i{LC!NewTitle(d \to t)}
	\post	\i{requestQ'}(t) = \langle\rangle
	\end{spec}
\showbeside
	\begin{spec}[\equiv]{\i{LR!NewTitle(d \to t)}}
		d:\g{D} \sep t:\g{T} \sep t \notin \i{Title}
	\post	t \in \i{Title' \cap\, ExColl'}\\
		\i{desc'}(t) =\,d \sep \i{requestQ'}(t) = \langle\rangle\\
		\i{nc'}(t) = \i{nQ'}(t) =\,0
	\end{spec}
\end{showspecs}

\medskip\noindent Further events will be required in this context to
update the pending requests for some particular title.
An event to allow a member to reserve a particular title:
\begin{showspecs}
	\begin{spec}{\i{LR!Reserve(t,m \to p)}}
		t: \i{Title} \sep m: \i{Member} \sep p: \p{POS}\\
		t \mapsto m \notin \i{reserve} \sep p = \i{nQ(t)} + 1
	\post	\i{requestQ'}(t) = (\i{requestQ}(t)) \langle m \rangle\\
		\i{nQ'}(t) =\,p\\
		t \mapsto m \in \i{reserve'}
	\end{spec}
\end{showspecs}

\medskip\noindent and an event to cancel such a reservation:
\begin{showspecs}
	\begin{spec}{\i{LR!Cancel(t,m \to p)}}
		t: \i{Title} \sep m: \i{Member} \sep p: \p{POS}\\
		t \mapsto m \in \i{reserve} \\
		Q_1 \langle m \rangle Q_2:= \i{requestQ}(t)\\
        p = \#\,Q_1 + 1
	\post	\i{requestQ'}(t) =\; Q_1 \langle\rangle\,Q_2\\
		\i{nQ'}(t) = \i{nQ}(t) - 1\\
		t \mapsto m \notin \i{reserve'}
	\end{spec}
\end{showspecs}
\noindent These two events preserve the \emph{relative}
ordering of all other pending requests whenever one for
title $t$ and member $m$ is either `reserved' or `cancelled';
they also give its new or previous position $p$ in the queue.

\medskip\noindent
\textbf{Exercise}\exer\\
Show that the assumption \emph{a member may have at most one request for every title outstanding at the same time}
is implicit from the invariant of class \i{LR}.

\medskip\noindent
\textbf{Exercise}\exer\\
Were we to remove a member, we'd need to consider their requests as well.
Would an event to remove a title need to consider requests for that title?
Develop specifications of these two events (remove member, remove title) in this class.

\newpage
\subsubsection*{Simple Library Class: $\i{LS}$}

We model a simple library class by \emph{composing} the loan (\i{LL}) and
reservation (\i{LR}) classes. The composed specification must be consistent:
\begin{showspecs}
	\begin{spec}{\i{LS}}
		\i{LL} \sep	\i{LR}\\[0.25ex]
		\i{loan \cap reserve} = \emptyset
	\end{spec}
\end{showspecs}
\begin{showspecs}
	\begin{spec}[\equiv]{\i{LS}}
		\i{Member}: \p{set}\;\g{M}\\
		\i{info}: \i{Member} \tfun \g{I}\\[0.5ex]
		\i{Title}: \p{set}\;\g{T}\\
		\i{desc}: \i{Title} \tfun \g{D}\\[0.5ex]
		\i{requestQ}: \i{Title}\tfun\p{inj}\cdot\p{seq}\;\i{Member}\\[0.5ex]
		\i{loan,reserve}: \i{Title} \prel \i{Member}\\
		\,\p{cf}\,\i{reserve} = \p{cod}\circ\i{requestQ}\\
		\i{loan \cap reserve} = \emptyset\\[0.5ex]
		\i{nc,na,nl,nQ}\!:\!\i{Title}\!\tfun\p{NAT}\\
		\i{InColl}:=\!\{t\!:\!\i{Title}\bullet\i{nc}(t) > 0\}\\
		\i{ExColl}\!:=\!\{t\!:\!\i{Title}\bullet\i{nc}(t) = 0\}\\
        \forall t: \i{Title}\, \bullet \\
        \hspace{0.5cm}\i{nc}(t) = \i{na}(t)+\i{nl}(t) \,\wedge\\
		\hspace{0.5cm}\i{nl}(t) = \#\{m:\i{Member}\bullet t\mapsto m\in \i{loan}\} \,\wedge\\
        \hspace{0.5cm}\{m:\i{Member}\bullet t\mapsto m\in \i{reserve}\} = \{m:\i{Member}\bullet m\in \p{cod}\,\i{requestQ}(t)\}\,\wedge\\
		\hspace{0.5cm}\i{nQ}(t) = \#\{m:\i{Member}\bullet t\mapsto m\in \i{reserve}\}\,\wedge\\
        \hspace{0.5cm}\i{nQ}(t) = \#\i{requestQ}(t)
	\post	\i{Member'} = \emptyset \sep \i{Title'} = \emptyset
	\end{spec}
\end{showspecs}
%
\medskip\noindent To maintain consistency, we have imposed a further constraint at this level:
\emph{for each title, a member may have at most one loan \emph{\,or} one reservation at the same time.}

\bigskip\noindent Most required queries are now direct promotions:
\begin{showspecs}
	\begin{spec}{\i{LS?showInfo(m \to i)}}
		\i{LM?showInfo(m \to i)}
	\end{spec}
\showbeside
	\begin{spec}[\equiv]{\i{LS?showInfo(m \to i)}}
		m: \i{Member} \sep i:= \i{info}(m)
	\end{spec}
\showmore
	\begin{spec}{\i{LS?showMembers(\to M)}}
		\i{LM?showMembers(\to M)}
	\end{spec}
\showbeside
	\begin{spec}[\equiv]{\i{LS?showMembers(\to M)}}
		M:= \i{info}
	\end{spec}
\showmore
	\begin{spec}{\i{LS?showDesc(t \to d)}}
		\i{LT?showDesc(t \to d)}
	\end{spec}
\showbeside
	\begin{spec}[\equiv]{\i{LS?showDesc(t \to d)}}
		d := \i{desc}(t)
	\end{spec}
\showmore
	\begin{spec}{\i{LS?showNoCopies(t \to n)}}
		\i{LC?showNoCopies(t \to n)}
	\end{spec}	
\showbeside
	\begin{spec}[\equiv]{\i{LS?showNoCopies(t \to n)}}
		n:=\i{nc}(t)
	\end{spec}
\showmore
	\begin{spec}{\i{LS?showLoans(t \to M)}}
		\i{LL?showLoans(t \to M)}
	\end{spec}
\showbeside
	\begin{spec}[\equiv]{\i{LS?showLoans(t \to M)}}
		M :=\{m:\i{Member}\,\bullet\, t\mapsto m\in \i{loan}\}
	\end{spec}
\showmore
	\begin{spec}{\i{LS?showRequests(t \to Q)}}
		\i{LR?showRequests(t \to Q)}
	\end{spec}
\showbeside
	\begin{spec}[\equiv]{\i{LS?showRequests(t \to Q)}}
		Q := \i{requestQ}(t)
	\end{spec}
\end{showspecs}

\medskip\noindent
Finally we'll update the \i{showCollection} query from the loan class to include the number of reservations for each title in the collection:
\begin{showspecs}
	\begin{spec}{\i{LS?showCollection(\to C)}}
		C: \i{Title}\pfun\p{POS}\!\times\p{NAT}\!\times\p{NAT}\!\times\g{D}\\
		C\!=\!\{(t,n,l,q,d):\i{InColl}\times\p{POS}\!\times\p{NAT}\!\times\p{NAT}\!\times\g{D}\;\bullet\\
        \hspace{1cm} n = \i{nc}(t)\wedge l = \i{nl}(t)\wedge q = \i{nQ}(t)\wedge d = \i{desc}(t)\}
	\end{spec}
\end{showspecs}

\bigskip\noindent Most events are also direct promotions:
\begin{showspecs}
	\begin{spec}{\i{LS!NewMember(i \to m)}}
		\i{LM!NewMember(i \to m)}
	\end{spec}
\showbeside
	\begin{spec}[\equiv]{\i{LS!NewMember(i \to m)}}
		i:\g{I} \sep m:\g{M} \sep m \notin \i{Member}
	\post	m \in \i{Member'} \sep \i{info'}(m) =\,i
	\end{spec}
\showmore
	\begin{spec}{\i{LS!UpdateInfo(m,i)}}
		\i{LM!UpdateInfo(m,i)}
	\end{spec}
\showbeside
	\begin{spec}[\equiv]{\i{LS!UpdateInfo(m,i)}}
		m:\i{Member} \sep i:\g{I} \sep i \neq \i{info}\;m
	\post	\i{info'}(m) =\,i
	\end{spec}
\showmore
	\begin{spec}{\i{LS!AddCopies(t,n)}}	
		\i{LL!AddCopies(t,n)}
	\end{spec}
\showbeside
	\begin{spec}[\equiv]{\i{LS!AddCopies(t,n)}}	
		t: \i{Title} \sep n: \p{POS}
	\post  \i{nc'}(t) = \i{nc}(t) + n\\
            \i{na'}(t) = \i{na}(t) + n
	\end{spec}
\showmore
	\begin{spec}{\i{LS!RemoveCopy(t)}}	
		\i{LL!RemoveCopy(t)}
	\end{spec}
\showbeside
	\begin{spec}[\equiv]{\i{LS!RemoveCopy(t)}}	
		t: \i{Title} \sep \i{nc}(t) \geq \i{na}(t) > 0
	\post	
		\i{nc'}(t) = \i{nc}(t) - 1\\
		\i{na'}(t) = \i{na}(t) - 1
	\end{spec}
\end{showspecs}

\newpage
\noindent We inherit versions of NewTitle from both loan and reservation classes and combine them:

\begin{showspecs}
\showmore\begin{spec}{\i{LS!NewTitle(d \to t)}}
		\i{LL!NewTitle(d \to t)}\\[0.25ex]
		\i{LR!NewTitle(d \to t)}
	\end{spec}
\showbeside
	\begin{spec}[\equiv]{\i{LS!NewTitle(d \to t)}}
		d:\g{D} \sep t:\g{T} \sep t \notin \i{Title}
	\post	t \in\!\i{Title'}\sep t \in\!\i{ExColl'}\\
        \i{desc'}(t) = d\\
		\i{nc'}(t) = \i{na'}(t) = \i{nl'}(t)=0\\[0.25ex]
        \i{nQ'}(t) = 0\\
		\i{requestQ'}(t) = \langle\rangle
	\end{spec}
\end{showspecs}

\bigskip\noindent  Stronger constraints are imposed for making a
reservation, to preserve the invariant:

\begin{showspecs}
	\begin{spec}{\i{LS!Reserve(t,m \to p)}}
		\i{LR!Reserve(t,m \to p)}\\[0.25ex]
		\,t \mapsto m \notin \i{loan}
	\end{spec}
\showbeside
	\begin{spec}[\equiv]{\i{LS!Reserve(t,m \to p)}}
		t: \i{Title} \sep m: \i{Member} \sep p: \p{POS}\\
		t \mapsto m \notin \i{reserve}\cup\i{loan}\\
		p = \i{nQ}(t) + 1
	\post	\i{requestQ'}(t) = (\i{requestQ}(t)) \langle m \rangle\\
		\i{nQ'}(t) =\,p\\
		t \mapsto m \in \i{reserve'}
	\end{spec}
\end{showspecs}

\bigskip\noindent The event \i{cancel} is just a promotion from the reservation class:
\begin{showspecs}
	\begin{spec}{\i{LS!Cancel(t,m \to p)}}
		\i{LR!Cancel(t,m \to p)}
	\end{spec}
\showbeside
	\begin{spec}{\i{LR!Cancel(t,m \to p)}}
		t: \i{Title} \sep m: \i{Member} \sep p: \p{POS}\\
		t \mapsto m \in \i{reserve} \\
		Q_1 \langle m \rangle Q_2:= \i{requestQ}(t)\\
        p = \#\,Q_1 + 1
	\post	\i{requestQ'}(t) =\; Q_1 \langle\rangle\,Q_2\\
		\i{nQ'}(t) = \i{nQ}(t) - 1\\
		t \mapsto m \notin \i{reserve'}
	\end{spec}
\end{showspecs}

\bigskip\noindent  The \i{LoanCopy} event becomes interesting when reservations are taken into account.
There are two cases: the member has reserved the title or not.
If not then the member can borrow a copy if there are more available copies than the number of reservations.
Otherwise, that is the member has reserved the title, the member can borrow a copy if
there are more available copies than the number of reservations prior to the member in the queue.
In this case, the member's reservation is cancelled when the copy is loaned.

\begin{showspecs}
	\begin{spec}{\i{LS!LoanCopy(t,m)}}
		\i{LL!LoanCopy(t,m)}\\
		\,n:\p{NAT} \sep \i{na}(t) > n\\
		\begin{alt}
			t\mapsto m \notin \i{reserve} \sep n = \i{nQ}(t)
		\end{alt}\\
		\begin{alt}
			\!\i{LR!Cancel}(t,m \to p)\\
			n = p-1
		\end{alt}
	\end{spec}
\showbeside
	\begin{spec}[\equiv]{\i{LS!LoanCopy(t,m)}}
		t:\i{Title} \sep m:\i{Member} \sep n: \p{NAT}\\
		t\mapsto m \notin \i{loan} \sep \i{na}(t) > n\\
		\begin{alt}
			t\mapsto m \notin \i{reserve} \sep n = \i{nQ}(t)
		\end{alt}\\
		\begin{alt}
			t\mapsto m \in \i{reserve}\\
			Q_1\langle\i{m}\rangle\,Q_2:= \i{requestQ}(t)\\
            n = \#\,Q_1
		\post
			\i{requestQ'}(t) =\;Q_1 \langle\rangle\,Q_2\\	
			\i{nQ'(t)} = \i{nQ(t)} - 1\\
			t \mapsto m \notin \i{reserve'}
		\end{alt}
	\post	t \mapsto m \in \i{loan'}\\
		\i{na'}(t) = \i{na}(t) - 1\\
		\i{nl'}(t) = \i{nl}(t) + 1
	\end{spec}
\end{showspecs}

\bigskip\noindent Finally, the event \i{Return} is a direct promotion from the loan class:
\begin{showspecs}
	\begin{spec}{\i{LS!Return(t,m)}}
		\i{LL!Return(t,m)}
	\end{spec}
\showbeside
	\begin{spec}[\equiv]{\i{LS!Return(t,m)}}
		t:\i{Title} \sep m: \i{Member}\\
		t\mapsto m \in \i{loan}\\
        \i{nl}(t) > 0
	\post	t\mapsto m \notin \i{loan'}\\
		\i{na'}(t) = \i{na}(t) + 1\\
		\i{nl'}(t) = \i{nl}(t) - 1
	\end{spec}
\end{showspecs}

\medskip\noindent
\textbf{Exercise}\exer\\
We have promoted or re-defined all queries at this level;
was it \emph{necessary} to specify them earlier?

\medskip\noindent
\textbf{Exercise}\exer\\
Explain the extended \i{LoanCopy} event as specified for
class \i{LS}, saying how its local variable $n$ corresponds
to the number of prior reservations in both sub-cases.

\medskip\noindent
\textbf{Exercise}\exer\\
Suggest ways in which the current Simple Library class can be extended or refined.
(This is a very open-ended question.)



\end{document} 